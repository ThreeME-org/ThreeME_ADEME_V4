% Options for packages loaded elsewhere
\PassOptionsToPackage{unicode}{hyperref}
\PassOptionsToPackage{hyphens}{url}
\PassOptionsToPackage{dvipsnames,svgnames,x11names}{xcolor}
%
\documentclass[
  letterpaper,
  DIV=11,
  numbers=noendperiod]{scrartcl}

\usepackage{amsmath,amssymb}
\usepackage{iftex}
\ifPDFTeX
  \usepackage[T1]{fontenc}
  \usepackage[utf8]{inputenc}
  \usepackage{textcomp} % provide euro and other symbols
\else % if luatex or xetex
  \usepackage{unicode-math}
  \defaultfontfeatures{Scale=MatchLowercase}
  \defaultfontfeatures[\rmfamily]{Ligatures=TeX,Scale=1}
\fi
\usepackage{lmodern}
\ifPDFTeX\else  
    % xetex/luatex font selection
\fi
% Use upquote if available, for straight quotes in verbatim environments
\IfFileExists{upquote.sty}{\usepackage{upquote}}{}
\IfFileExists{microtype.sty}{% use microtype if available
  \usepackage[]{microtype}
  \UseMicrotypeSet[protrusion]{basicmath} % disable protrusion for tt fonts
}{}
\makeatletter
\@ifundefined{KOMAClassName}{% if non-KOMA class
  \IfFileExists{parskip.sty}{%
    \usepackage{parskip}
  }{% else
    \setlength{\parindent}{0pt}
    \setlength{\parskip}{6pt plus 2pt minus 1pt}}
}{% if KOMA class
  \KOMAoptions{parskip=half}}
\makeatother
\usepackage{xcolor}
\setlength{\emergencystretch}{3em} % prevent overfull lines
\setcounter{secnumdepth}{-\maxdimen} % remove section numbering
% Make \paragraph and \subparagraph free-standing
\ifx\paragraph\undefined\else
  \let\oldparagraph\paragraph
  \renewcommand{\paragraph}[1]{\oldparagraph{#1}\mbox{}}
\fi
\ifx\subparagraph\undefined\else
  \let\oldsubparagraph\subparagraph
  \renewcommand{\subparagraph}[1]{\oldsubparagraph{#1}\mbox{}}
\fi


\providecommand{\tightlist}{%
  \setlength{\itemsep}{0pt}\setlength{\parskip}{0pt}}\usepackage{longtable,booktabs,array}
\usepackage{calc} % for calculating minipage widths
% Correct order of tables after \paragraph or \subparagraph
\usepackage{etoolbox}
\makeatletter
\patchcmd\longtable{\par}{\if@noskipsec\mbox{}\fi\par}{}{}
\makeatother
% Allow footnotes in longtable head/foot
\IfFileExists{footnotehyper.sty}{\usepackage{footnotehyper}}{\usepackage{footnote}}
\makesavenoteenv{longtable}
\usepackage{graphicx}
\makeatletter
\def\maxwidth{\ifdim\Gin@nat@width>\linewidth\linewidth\else\Gin@nat@width\fi}
\def\maxheight{\ifdim\Gin@nat@height>\textheight\textheight\else\Gin@nat@height\fi}
\makeatother
% Scale images if necessary, so that they will not overflow the page
% margins by default, and it is still possible to overwrite the defaults
% using explicit options in \includegraphics[width, height, ...]{}
\setkeys{Gin}{width=\maxwidth,height=\maxheight,keepaspectratio}
% Set default figure placement to htbp
\makeatletter
\def\fps@figure{htbp}
\makeatother

\KOMAoption{captions}{tableheading}
\makeatletter
\makeatother
\makeatletter
\makeatother
\makeatletter
\@ifpackageloaded{caption}{}{\usepackage{caption}}
\AtBeginDocument{%
\ifdefined\contentsname
  \renewcommand*\contentsname{Table des matières}
\else
  \newcommand\contentsname{Table des matières}
\fi
\ifdefined\listfigurename
  \renewcommand*\listfigurename{Liste des Figures}
\else
  \newcommand\listfigurename{Liste des Figures}
\fi
\ifdefined\listtablename
  \renewcommand*\listtablename{Liste des Tables}
\else
  \newcommand\listtablename{Liste des Tables}
\fi
\ifdefined\figurename
  \renewcommand*\figurename{Figure}
\else
  \newcommand\figurename{Figure}
\fi
\ifdefined\tablename
  \renewcommand*\tablename{Table}
\else
  \newcommand\tablename{Table}
\fi
}
\@ifpackageloaded{float}{}{\usepackage{float}}
\floatstyle{ruled}
\@ifundefined{c@chapter}{\newfloat{codelisting}{h}{lop}}{\newfloat{codelisting}{h}{lop}[chapter]}
\floatname{codelisting}{Listing}
\newcommand*\listoflistings{\listof{codelisting}{Liste des Listings}}
\makeatother
\makeatletter
\@ifpackageloaded{caption}{}{\usepackage{caption}}
\@ifpackageloaded{subcaption}{}{\usepackage{subcaption}}
\makeatother
\makeatletter
\@ifpackageloaded{tcolorbox}{}{\usepackage[skins,breakable]{tcolorbox}}
\makeatother
\makeatletter
\@ifundefined{shadecolor}{\definecolor{shadecolor}{rgb}{.97, .97, .97}}
\makeatother
\makeatletter
\makeatother
\makeatletter
\makeatother
\ifLuaTeX
\usepackage[bidi=basic]{babel}
\else
\usepackage[bidi=default]{babel}
\fi
\babelprovide[main,import]{french}
% get rid of language-specific shorthands (see #6817):
\let\LanguageShortHands\languageshorthands
\def\languageshorthands#1{}
\ifLuaTeX
  \usepackage{selnolig}  % disable illegal ligatures
\fi
\usepackage[]{natbib}
\bibliographystyle{plainnat}
\IfFileExists{bookmark.sty}{\usepackage{bookmark}}{\usepackage{hyperref}}
\IfFileExists{xurl.sty}{\usepackage{xurl}}{} % add URL line breaks if available
\urlstyle{same} % disable monospaced font for URLs
\hypersetup{
  pdftitle={Macroeconomic impact from implementing energy shield tariffs: an assesment trough the use of a CGE model},
  pdfauthor={Paul Malliet, Anissa Saumtally},
  pdflang={fr},
  colorlinks=true,
  linkcolor={blue},
  filecolor={Maroon},
  citecolor={Blue},
  urlcolor={Blue},
  pdfcreator={LaTeX via pandoc}}

\title{Macroeconomic impact from implementing energy shield tariffs: an
assesment trough the use of a CGE model}
\author{Paul Malliet, Anissa Saumtally}
\date{2023-06-09}

\begin{document}
\maketitle
\ifdefined\Shaded\renewenvironment{Shaded}{\begin{tcolorbox}[boxrule=0pt, interior hidden, breakable, enhanced, borderline west={3pt}{0pt}{shadecolor}, sharp corners, frame hidden]}{\end{tcolorbox}}\fi

\renewcommand*\contentsname{Table des matières}
{
\hypersetup{linkcolor=}
\setcounter{tocdepth}{3}
\tableofcontents
}
\hypertarget{introduction}{%
\section{Introduction}\label{introduction}}

Russia's invasion of Ukraine in February 2022 triggered a major energy
crisis for European Union countries that started in September 2021 in
the wake of the post-COVID recovery of the international demand .
Between December 2020 and December 2021, the import price for energy in
the euro area already more than doubled, spurring inflation across
European countries. Regarding the french situation, From 2021Q2 and
2022Q2, energy prices' increase contributed to 3.1 points of percentages
(pp) out of a total of 5.3\% of inflation in France
\citep{bourgeois_alexandre_flambee_2022}.

If the reaction was quick to condemn Russia, through the ban on Russian
trade, the dramatic rise of energy prices has been conducted to well as
intervention policies to moderate the impact of this rise on households
and businesses \citep{cae2023b} as is the case for France with the use
of energy shield tariff from October 2021. The operating principle of
the energy shield is based on a state subsidy between a capped consumer
price and the supplier's price according to market conditions. Thus,
while it guarantees the targeted level of inflation, its cost to the
public authorities depends first and foremost on market prices. Among
the measures implemented by European countries, the choice made by
France of a policy of maintaining price levels is quite distinctive
\citep{Bruegel} since it is the sole country to have adopted these
measures, transfers and tax reductions being more often used by the
other member states.

Forecasting in the PLF issue an estimated cost for the policy for 2022
and 2023 to respectively XX and XX

This study aims at understanding the macroeconomic effects from energy
shield tariffs through an explicit representation of the price structure
from wholesale market prices to consumer prices. In the first section,
we present the methodology used to assess the effects from implementing
shield tariff, section 2 presents the data, Section 3 we present the
results, that we discussed in the section 4. Section 5 concludes.

\hypertarget{modeling-framework-and-scenarios}{%
\section{Modeling Framework and
Scenarios}\label{modeling-framework-and-scenarios}}

\hypertarget{threeme-model}{%
\subsection{ThreeME model}\label{threeme-model}}

ThreeME is a country-level open source Computable General Equilibrium
model (CGE) originally developed to support policymakers in the design
and evaluation of decarbonization pathways in France (Callonnec
et~al.~2013a, b, 2016).3 Since its first release, it has also been
adapted to Mexico (Landa et~al.~2016), Indonesia (Malliet et~al.~2016)
and the Netherlands (Bulavskaya and Reynès 2018). ThreeME is
specifically designed to evaluate the short-, medium- and long-term
impact of environmental and energy policies at the macroeconomic and
sectoral levels. To this end, the model combines several important
features:

\begin{itemize}
\tightlist
\item
  Its sectoral disaggregation allows for analyzing the transfer of
  activities from one sector to another, particularly in terms of
  employment, investment, energy consumption or balance of trade.
\item
  The highly detailed representation of energy flows through the economy
  allows for analyzing the consumption behavior of economic agents with
  respect to energy. Sectors can arbitrage between capital and energy
  when the relative price of energy increases, and substitute between
  energy vectors. Consumers can substitute between energy vectors,
  transportation modes or consumption goods.
\end{itemize}

As a CGE model, ThreeME fully considers feedbacks between supply and
demand (see \citet{fig1}). Demand (consumption and investment) drives
the supply (production). Symmetrically supply drives demand through the
incomes generated by the production factors (labor, capital, energy
products and materials). Compared to bottom-up energy models such as
MARKAL \citep{markal1981} or TIMES {[}loulou2005documentation{]},
ThreeME goes beyond the mere description of the sectoral and
technological dimensions by integrating these within a comprehensive
macroeconomic model. ThreeME is a neo-Keynesian model. Compared to
standard Walrasian-type CGEs that are largely supply driven, prices do
not adjust instantaneously to clear markets. Instead the model is
dynamic, and prices and quantities adjust slowly. Producers adjusts
their supply to the demand. This has the advantage to allow for
situations of market disequilibria (in particular the presence of
involuntary unemployment). This framework is particularly well suited
for policy analysis. In addition to providing information about the long
term, it allows for analyzing transition phases over the short and
medium terms, which is especially relevant when assessing the
implementation of climate and energy policies.

\begin{figure}

{\centering \includegraphics{figures/Fig1.pdf}

}

\caption{``Schematic representation of the model ThreeME''}

\end{figure}

Fig 1 demonstrates the methodological framework of ThreeME. This model
maximizes the utility of each agent in period t subject to several
constraints, such as market clearing (e.g., demand is equal to supply).
The model is recursive dynamic (i.e., myopic), which means it first
optimizes period \(t\) and then uses the endogenous results (e.g.,
prices, wages, and production levels) for optimizing the next
period(i.e., \(t + 1\)). After the model optimizes the last period
(determined by the user), it provides the projection of the endogenous
parameters, such as prices, household income, GDP, and employment rate,
over the whole horizon. Moreover, ThreeME requires several exogenous
parameters: the social accounting matrix (SAM) of the base year,
population growth forecast, economic growth forecast, and substitution
elasticities. SAM is a comprehensive and economy-wide database recording
data about all transactions between economic agents in a specific
economy for a specific period {[}kehoe1996social{]}. The population and
economic growth forecasts determine labor availability and productivity
projection. Elasticities define the substitution proportion of
production factors in production functions. In a CES function, the
substitution between production factors can either follow the linear,
fixed-proportion (i.e., Leontief) or CobbDouglas production functions.
The linear production function represents a production process in which
the inputs are perfect substitutes (e. g., labor can be substituted
completely with capital). The fixed proportion production function
reflects a production process in which the inputs are required in fixed
proportions. In the Cobb-Douglas production function, the inputs can be
substituted,if not perfectly. ThreeME assumes a nested CES function
\citep{reynes2019cobb} to describe the substitution between production
factors. This CES production function requires four inputs, KLEM,
capital (K), labor (L), energy (E), and material (M). The production
factors (KLEM) can be substituted with each other. The Elasticity of
Substitution (ES) parameters determines the substitution level between
each input. Each pair (i.e., K-E, KE-L, KEL-M) has its own ES, which is
explained further in the description of the model
\citep{reynes2021threeme}. An essential characteristic of a standard
neoKeynesian macroeconomic AS-AD (aggregated supply and demand) model is
that demand determines supply. The demand comprises (intermediate and
final) consumption, investment and export whereas the supply comes from
imports and domestic production. As feedback with eventually some lags,
supply affects demand through several mechanisms. The level of
production determines the quantity of inputs used by the firms and thus
the quantity of their intermediate consumption and investment which are
two components of the demand. It determines the level of employment as
well and consequently the household final consumption. Another effect of
employment on demand goes through the wage settingvia the
unemploymentrate which is also determinedby the active population. The
active population is mainly determined by exogenous factors such as
demography but also by endogenous factors: because of discouraged worker
effects, the unemployment rate may affect the labor participation rate
and thus the active population

\hypertarget{integration-of-the-shield-tariff-mechanism}{%
\subsection{Integration of the Shield tariff
mechanism}\label{integration-of-the-shield-tariff-mechanism}}

The introduction of the Energy Shield Tariff (EST) into the model
dynamics implies to change the equations setting the relation between a
wholesale gross price (equivalently producer price) and the final
consumer price. In the generic model, the growth rate of consumer price
depends on the growth rate of production prices, the applied margin and
taxes such as:

\[ 
\Delta(\log(P^{CH} = \Delta(\log(P^Y + MARGIN + NTAX))
\]

\hypertarget{data}{%
\section{Data}\label{data}}

\hypertarget{the-french-system-for-regulation-of-energy-prices-titre-nul-a-changer}{%
\subsection{The French system for regulation of energy prices {[}titre
nul a
changer{]}}\label{the-french-system-for-regulation-of-energy-prices-titre-nul-a-changer}}

Despite being an open market, Electricity and gas consumers in France
have the option between a market price and a regulated price. Previously
a public monopoly, the supply of gas and electricity opened to newcomers
in xxxx, making the market supposeldy competitive. The regulated price
option made available to consumers is sold by the historical supplier
only in the obj. Market options on the hand tend to be

\begin{itemize}
\tightlist
\item
\end{itemize}

-As a result, energy consumer prices in France fluctuations tend to be
overall quite moderate. The passthrough of the variations of wholesale
prices is in fact one the lowest in Europe, especially for electricity

\hypertarget{regulating-energy-prices-in-times-of-crisis-the-energy-price-shield}{%
\subsection{Regulating energy prices in times of crisis : the energy
price
shield}\label{regulating-energy-prices-in-times-of-crisis-the-energy-price-shield}}

Towards the second half of 2021, wholesale gas prices experienced some
high peaks (variations of magitude x 10 on daily averages) after
remaining at record low points in 2020 in the midst of the Covid-19
pandemic. As economies were restarting fast especially in Asia, and
Europe was experiencing a cold winter, the tension of the demand on
natural gas lead to a first severe increase in wholesale prices. This
energy inflation crisis only worsened in 2022 due to the second Russian
invasion of Ukraine on February 24th 2022. As an embargo on Russian
gas,was agreed upon by the EU, albeit very difficultly so due to the
high dependence of some member states, the cost of alternatives (mainly
liquefied gas (LPG)) increased all the more. On the Dutch TTF market,
the reference point for wholesale prices in Europe, the intraday trading
prices reached temporary highs above the 1000 euros mark

For France, as the gas regulated prices are evaluated monthly, the usual
method of calculating prices by the CRE would have integrated these
variations, and consumer prices would have followed suit. This marked
the beginning of the government's intervention for gas prices. The
initial measure of the \textbf{bouclier tarifaire}, or price shield,
consisted in freezing the regulated tariff for household consumers on
these types of contracts starting the end of the year 2021 and to be
continued in 2022. For electricity prices, unaffected in 2021, the
regulated price increase in 2022 would be limited to 4 \% instead of an
estimated increase of 30 \%. For 2023, this price increase cap was fixed
at 15 \% for both household gas\footnote{For gas, this price increase is
  initially limited to the first half of the year. Given the current
  wholesale gas prices, it is unlikely to be extended to the rest of the
  year.} and electricity. In practice, this measure aims to cap the
price, all taxes included, of the kilowatt-hour. There is in theory no
condition made on the quantity consumed. It should be noted however that
the government also ran a campaign in parallel encouraging energy
savings during these time, citing risks of shortages. In 2022, the
consumption of electricity and gas did decrease despite the relatively
low price surge

In order to limit the impact on suppliers, this cap is reached first by
a removal of some taxes applicable to the consumer for those products
(the TICFE {[}acronym def{]}). To complete this, the CRE would then
calculate a price that will reach the target by reducing the energy
supplier compensation part of the commodity price. They also provide a
theoretical price that would have been implemented without a price cap.
The government will then subsidise the difference directly to supplier.
When evaluating the total cost of this policy, it is the tax-rebate and
the subsidy that are added. Figure xx below summarises the average
theoretical and applicable prices; on this graph, the difference between
dotted and solid lines represent the cost of the measure (not including
tax rebates) for the governement.

The latest estimates by the French government of this policy alone in
the Stability Program evaluated the cost in 2022 at xx billions euros
for gas and xx billions euros for electricity. For 2023 as wholesale
pricing are decreasing sharply (in may they reached levels below the
2021 average), the cost of the policy is so far estimated at xx for
electricity and xx for gas. We note that the high estimated cost for
electricity is mostly due to the design of the pricing system. As
mentioned, regulated electricity prices are calculated yearly by the
CRE, with minor adjustments at the second half of the year. The bulk of
the theoretical price increase (which drives the cost of the measure) is
likely due to the supplier compensation that integrates a make-up for
the losses in the previous year due to unforeseen market prices
increase.

Cost-mitigation for consumers also included some targeted measures such
as direct-to consumer subsidies for low-income households (\emph{chèque
énergie}), the cost of which is relatively minimal in comparison with
the price shield measure. In this paper we also do not consider the
increase in petrol prices, though it should be noted that a temporary
price reduction on petrol subsidised by the governement was also put in
place in 2022, which benefited all consumers for any amount consumed.

\hypertarget{calibrating-threeme-and-integrating-the-cre-data}{%
\subsection{Calibrating ThreeME and integrating the CRE
data}\label{calibrating-threeme-and-integrating-the-cre-data}}

ThreeME model for France has been calibraed using national accounts data
available through Eurostat. The data version used in this paper is based
at the year 2015. After the base year, the only shock integrated is
world wide energy prices starting 2021 to represent and isolate the
observed price dynamics and allow for an analysis independent of any
other variations in the economy. In order to model the energy consumer
price energy in France, an additional modification is made on consumer
prices to reflect the price regulation structure mentioned above. All
other variables thus evolve using ThreeME's steady state specifications
for France {[}ref{]}.

\hypertarget{results}{%
\section{Results}\label{results}}

\hypertarget{discussion}{%
\section{Discussion}\label{discussion}}

\hypertarget{conclusion}{%
\section{Conclusion}\label{conclusion}}



\end{document}
